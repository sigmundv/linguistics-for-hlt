\documentclass[12pt,%
    %anonym,
    times,
    %referee,
    % doublespacing,   
]{lin-v2/lin}
\usepackage[utf8]{inputenc}
\usepackage[T1]{fontenc}

\usepackage{amssymb}
\usepackage{tipa}

\usepackage[noglossaries]{leipzig}
\usepackage{gb4e}
\noautomath

\usepackage{rrgtrees}
\usepackage{pst-node}
\usepackage{pst-tree}

\begin{document}

\leftrunning{}  % Short author list

\rightrunning{} % Short title

\title{A Comparison of Language Complexity in Faroese and Icelandic}

\author[1]{\givenname{Sigmund} \surname{Vestergaard} -- B00108126}

\address[1]
{%
  \inst{Institute of Technology Blanchardstown}, % Institution name should be in \inst
  \addr{Blanchardstown Road North}, % Street 
  \addr{Dublin 15}, % Postcode etc
  \cnty{Ireland}  % Country
  \email{B00108126@student.itb.ie} % email
}

\maketitle
%\tableofcontents*

\begin{abstract}
    We examine the complexity of Icelandic compared to Faroese, which are two closely related North Germanic languages.
    It is found that Icelandic is more complex than Faroese, and evidence is presented that shows
    that while inflection of nouns, agreement of case, number, and gender is similar for both languages,
    verb inflection in Icelandic is much richer than it is in Faroese. As an example we can mentioned
    case patterns for objects of ditransitive verbs, where Icelandic has a significant number of pattern
    combinations while Faroese predominantly has the dative--accusative pattern, and in some cases
    accusative--accusative.
    \keywords{Complexity, Faroese, Icelandic}
\end{abstract}

\section{Introduction}

In this paper we introduce the two Germanic languages of Icelandic and Faroese and present aspects of their morphology
and syntax, respectively, with a view to comparing the complexity of the two.

% will introduce the concept of \emph{language complexity} and discuss different views on language complexity.
% We ask whether all languages are equally complex and how we would go about defining linguistic complexity of a language.
% After we have set the scene we will apply the theory to the Norse languages of Faroese and Icelandic, respectively,
% and compare the complexity of these two languages.

First we briefly introduce both languages, putting them in contex. Then we move on to analysis of Icelandic morphology and syntax.
Finally we do the same for Faroese while referring back to the analysis of Icelandic, which should give us an idea of
the similarities and differences, respectively, between the two languages.
The examples and accompanying analysis for Faroese are drawn from~\citep{faroese},
while for Icelandic we mostly draw on~\citep{icelandic}.\\

% This introductory section will first touch upon the concept and present different viewpoints and types of complexity.
% Then we will briefly put Icelandic and Faroese, respectively, in context.

The hypothesis we wish to explore in this paper is the following:
\begin{exe}
    \ex
    Icelandic is essentially more complex that Faroese,
    because Faroese has developed further away from Old Norse than Icelandic has.
\end{exe}


% \subsection{Language Complexity}




\subsection{Icelandic}

Icelandic developed from the Old Norse spoken by the settlers that arrived from Norway
in the late ninth and early tenth century. Today it is primarily spoken by around 335,000
inhabitants in Iceland and is most closely related to the other Nordic languages, Faroese, Norwegian, Swedish, and Danish,
but more so to Faroese and the dialects of southwestern Norway than the others.

Normally the history of Icelandic is divided into the Old Icelandic period, before 1540,
and the Modern Icelandic period, after 1540. The dividing line of 1540 is the year
the first Icelandic translation of the New Testament was published.

When Modern Icelandic, which is what we will concern ourselves with in this paper,
is compared with other Nordic languages on one hand, and Old Norse on the other,
it is evident that Icelandic has changed less than the other
Modern Scandinavian languages with respect to morphology and
syntax~\citep{icelandic, germanicIcelandic}.


\subsection{Faroese}

Faroese is derived from the Norse language of the primarily Norwegian settlers
who moved to the islands in the ninth century. Due to a lack of Viking Age
and medieval sources little is known about the development of Faroese into
the language we know today. Nothing of substance was written in Faroese
until the 1770s, by which time most the features of the modern language
must have developed.

From the Reformation in 1539 until 1948 Faroese had no official status,
which has had several consequences for Faroese. The spoken language is
mostly dialectically based and its lexicon is heavily influenced by
Danish. On the other hand the written language is homogeneous and very
puristic. The orthography was only established in the middle of the nineteenth century,
and owes much to Old Norse and some to Icelandic. Neologisms are widespread
in Faroese, as they are in Icelandic, but many of them are only used in
writing, with the spoken language preferring variants derived from
Danish~\citep{germanicFaroese}.


\section{Icelandic}

In this section we go through the main grammatical categories, nominal and verbal,
and give a brief overview.

\subsection{Nominal categories}

Icelandic has a three-valued gender system, masculine, feminine, and neuter.
Grammatical gender is not necessarily inferred from the sex of the referents.
Most nouns referring to females are feminine, for example, but it is also possible to find
masculine and neuter words referring to females. It is also possible for
words referring to things and concepts to be either masculine, feminine, or neuter.
We list some examples here~\citep{icelandic}:
\begin{exe}
    \ex \begin{xlist}
        \item strákur (m) 'boy', stóll (m) 'chair', svanni (m) 'woman (poetic)'
        \item stelpa (f) 'girl', mynd (f) 'picture', hetja (f) 'hero'
        \item barn (n) 'child', borð (n) 'table', fljóð (n) 'woman (poetic)', skáld (n) 'poet'
    \end{xlist}
\end{exe}

Nominal categories, such as nouns, adjectives, articles and pronouns, have four cases,
nominative (N), accusative (A), dative (D), and genitive (G), and two numbers, singular (sg.) and plural (pl.). 
The inflection of nouns varies according to gender and inflectional class of the noun.
Both attributive and predictive adjectives agree with gender, case and number of the noun they modify.
Let's look at some examples~\citep{icelandic}:
\begin{exe}
    \ex \begin{xlist}
        \item \gll gul\emph{ur} hestur gul mynd\\
                   yellow.\N.\Sg.\M{} horse.\N.\Sg.\M{} yellow.\N.\Sg.\F{} picture.\N.\Sg.\F\\
              \gll gul\emph{t} borð gul\emph{ar} myndir\\
                   yellow.\N.\Sg.\N{} table.\N.\Sg.\N{} yellow.\N.\Pl.\F{} pictures.\N.\Pl.\F\\
        \item \gll Ég sá gul\emph{a} hænu\\
                   I saw yellow.\Acc.\Sg.\F{} hen.\Acc.\Sg.\F\\
                   \trans 'I saw a yellow hen'
        \item \gll Þessar hænur eru gul\emph{ar}\\
                   These hens.\N.\Pl.\F{} are yellow.\N.\Pl.\F\\
        \end{xlist}
\end{exe}

Icelandic does not have an indefinite artice, and the definite article is suffixed to nouns
according to their gender, noun, and case, as illustrated here~\citep{icelandic}:
\begin{exe}
    \ex
        \gll    hest-ur-\emph{inn} mynd-\emph{in} borð-\emph{ið}\\
                horse.\Nom.\Sg.\M.\Def{} picture.\Nom.\Sg.\F.\Def{} table.\Nom.\Sg.\N.\Def\\
        \trans  'the horse' ; 'the picture' ; 'the table'
\end{exe}


\subsection{Verbal categories}

Finite verbs in Icelandic agree with nominal subjects in person and number. Just as we saw with nouns,
the morphological markers for person and number appear to be fused; or at least very difficult to separate,
although it can be argued that person and number are distinct syntactic categories
in Icelandic after all~\citep[8]{icelandic}. Here are examples of inflection of two
different verbs, \emph{horfa} (look) and \emph{bíta} (bite), where the former is an example
of a weak (regular) verb and the latter is an example of a strong (irregular) verb~\citep{icelandic}:
\begin{exe}
    \ex \begin{xlist}
        \item \gll ég horf-i ég horf-ð-i\\
                   I look.\First\Sg.\Prs.\Ind{} I look.\First.\Sg.\Pst.\Ind\\
              \trans 'I look' 'I looked'
        \item \gll þú horf-ir þú horf-ð-ir\\
                   you look.\Second.\Sg.\Prs.\Ind{} you look.\Second.\Sg.\Pst.\Ind\\
              \trans 'you look' 'you looked'
        \item \gll hann horf-ir hann horf-ð-i\\
                   he look.\Third.\Sg.\Prs.\Ind{} he look.\Third.\Sg.\Pst.\Ind\\
                   \trans 'he looks' 'he looked'\\
        \item \gll við horf-um við horf-ð-um\\
                   we look.\First\Sg.\Prs.\Ind{} we look.\First.\Sg.\Pl.\Ind\\
              \trans 'we look' 'we looked'
        \item \gll þið horf-ið þið horf-ð-uð\\
                   you.\Pl{} look.\Second.\Sg.\Prs.\Ind{} you.\Pl{} look.\Second.\Pl.\Pst.\Ind\\
              \trans 'you look' 'you looked'
        \item \gll þeir horf-a þeir horf-ð-u\\
                   they.\M{} look.\Third.\Sg.\Prs.\Ind{} they.\M{} look.\Third.\Pl.\Pst.\Ind\\
              \trans 'they look' 'they looked'\\
        \end{xlist}
\end{exe}
\begin{exe}
    \ex \begin{xlist}
        \item \gll ég bít ég beit\\
                   I bite.\First\Sg.\Prs.\Ind{} I bite.\First.\Sg.\Pst.\Ind\\
              \trans 'I bite' 'I bit'
        \item \gll þú bít-ur þú bei-st\\
                   you bite.\Second.\Sg.\Prs.\Ind{} you bite.\Second.\Sg.\Pst.\Ind\\
              \trans 'you bite' 'you bit'
        \item \gll hann bít-ur hann beit\\
                   he bite.\Third.\Sg.\Prs.\Ind{} he bite.\Third.\Sg.\Pst.\Ind\\
              \trans 'he bites' 'he bit'
        \item \gll við bít-um við bit-um\\
                   I bite.\First\Pl.\Prs.\Ind{} I bite.\First.\Pl.\Pst.\Ind\\
              \trans 'I bite' 'I bit'
        \item \gll þið bít-ið þið bit-uð\\
                   you bite.\Second.\Pl.\Prs.\Ind{} you bite.\Second.\Pl.\Pst.\Ind\\
              \trans 'you bite' 'you bit'
        \item \gll þeir bít-a þeir bit-u\\
                   he bite.\Third.\Pl.\Prs.\Ind{} he bite.\Third.\Pl.\Pst.\Ind\\
              \trans 'he bites' 'he bit'
        \end{xlist}
\end{exe}

Icelandic has two distinct tenses, an unmarked \emph{present} tense and \emph{past} tense.
As is usually the case for Germanic languages, the past tense of weak verbs is marked with a dental suffix
while strong verbs show various (systematic but unpredictable) vowel changes (ablaut patterns).
The rich agreement morphology illustrated in examples~(4) and~(5) above is one of the main
differences between Icelandic and other Scandinavian languages, and it should be noted
that it is found both in indicative (as shown in the examples above) and subjunctive mood
(as shown in the examples below)~\citep{icelandic}:
\begin{exe}
    \ex \begin{xlist}
        \item \gll ég horf-i ég horf-ð-i\\
                   I look.\First\Sg.\Prs.\Subj{} I look.\First.\Sg.\Pst.\Subj\\
              \trans 'I look' 'I looked'
        \item \gll þú horf-ir þú horf-ð-ir\\
                   you look.\Second.\Sg.\Prs.\Subj{} you look.\Second.\Sg.\Pst.\Subj\\
              \trans 'you look' 'you looked'
        \item \gll hann horf-i hann horf-ð-i\\
                   he look.\Third.\Sg.\Prs.\Subj{} he look.\Third.\Sg.\Pst.\Subj\\
                   \trans 'he looks' 'he looked'\\
        \item \gll við horf-um við horf-ð-um\\
                   we look.\First\Sg.\Prs.\Subj{} we look.\First.\Sg.\Pl.\Subj\\
              \trans 'we look' 'we looked'
        \item \gll þið horf-ið þið horf-ð-uð\\
                   you.\Pl{} look.\Second.\Sg.\Prs.\Subj{} you.\Pl{} look.\Second.\Pl.\Pst.\Subj\\
              \trans 'you look' 'you looked'
        \item \gll þeir horf-i þeir horf-ð-u\\
                   they.\M{} look.\Third.\Sg.\Prs.\Subj{} they.\M{} look.\Third.\Pl.\Pst.\Subj\\
              \trans 'they look' 'they looked'\\
        \end{xlist}
\end{exe}
\begin{exe}
    \ex \begin{xlist}
        \item \gll ég bít-i ég bit-i\\
                   I bite.\First\Sg.\Prs.\Subj{} I bite.\First.\Sg.\Pst.\Subj\\
              \trans 'I bite' 'I bit'
        \item \gll þú bít-ir þú bit-ir\\
                   you bite.\Second.\Sg.\Prs.\Subj{} you bite.\Second.\Sg.\Pst.\Subj\\
              \trans 'you bite' 'you bit'
        \item \gll hann bít-i hann bit-i\\
                   he bite.\Third.\Sg.\Prs.\Subj{} he bite.\Third.\Sg.\Pst.\Subj\\
              \trans 'he bites' 'he bit'
        \item \gll við bít-um við bit-um\\
                   I bite.\First\Pl.\Prs.\Subj{} I bite.\First.\Pl.\Pst.\Subj\\
              \trans 'I bite' 'I bit'
        \item \gll þið bít-ið þið bit-uð\\
                   you bite.\Second.\Pl.\Prs.\Subj{} you bite.\Second.\Pl.\Pst.\Subj\\
              \trans 'you bite' 'you bit'
        \item \gll þeir bít-i þeir bit-u\\
                   he bite.\Third.\Pl.\Prs.\Subj{} he bite.\Third.\Pl.\Pst.\Subj\\
              \trans 'he bites' 'he bit'
        \end{xlist}
\end{exe}

The non-finite verb forms in Modern Icelandic are 1) \emph{infinitive} and 2) the two \emph{participles},
\emph{past} and \emph{present}. Infinitive typically ends with \emph{-a} in Icelandis,
as we can see from this comparison with imperative:
\begin{exe}
    \ex
    \gll infinitives: tala horf-a dæm-a bít-a\\
    imperatives: tala horf dæm bít\\
    \trans {~~~~~~~~~~~~~~~~~~~} 'talk' 'look' 'judge' 'bite'
\end{exe}

Present participle is formed by adding \emph{-(a)ndi} to the stem of the verb,
e.g. \emph{sof\emph{andi}} 'sleeping', \emph{gang\emph{andi}} 'walking', and does not inflect
in Modern Icelandic.

Past participle usually ends in \emph{-ur} or \emph{-inn} and inflects in gender, number, and case, as illustrated here:
\begin{exe}
    \ex
    \gll dæm-d-ur dæm-d dæm-t\\
    judge.\First\Nom.\Sg.\M{} judge.\First\Nom.\Sg.\F{} judge.\First\Nom.\Sg.\N\\
    \gll dæm-d-an dæm-d-a dæm-t\\
    judge.\First\Acc.\Sg.\M{} judge.\First\Acc.\Sg.\F{} judge.\First\Acc.\Sg.\N\\
    \gll dæm-d-ir dæm-d-ar dæm-d\\
    judge.\First\Nom.\Pl.\M{} judge.\First\Nom.\Pl.\F{} judge.\First\Nom.\Pl.\N\\
    \gll dæm-d-a dæm-d-ar dæm-d\\
    judge.\First\Acc.\Pl.\M{} judge.\First\Acc.\Pl.\F{} judge.\First\Acc.\Pl.\N\\
    \trans 'judged'
\end{exe}

The past participle is used in passive constructions, as shown here:
\begin{exe}
    \ex
    \gll Hundurinn var \emph{bitinn}\\
    dog.\Third\Nom.\Sg.\M.\Def{} be.\Aux{} bite.\Third\Nom.\Sg.\M.\Pst.\Ptcp{}\\
    \trans 'The dog was bitten'
    \ex
    \gll Bækurnar voru \emph{lesnar}\\
    book.\Third\Nom.\Pl.\F.\Def{} be.\Aux{} read.\Third\Nom.\Pl.\F.\Pst.\Ptcp\\
\end{exe}

Above we have seen examples with the auxiliary verb \emph{vera} 'be'.
Auxiliary verbs in Icelandic do not form a separate inflectional class,
but show the same rich agreement morphology and inflection for tense
as other verbs, hence can only be defined as 'the class of verbs that that are used to systematically
express grammatical categories' such as passive, perfect, progressive, and various modal constructions~\citep[10]{icelandic}.

As we have seen before, the passive is formed by the auxiliaries \emph{vera} 'be' and \emph{verða} 'become' plus a past participle.
We have also shown above that the passive auxiliary agrees with a nominative subject in person and number, and with the participle in
number and gender. The agent of a passive can be expressed with the preposition \emph{af} 'by' plus dative, but normally is is left
unexpressed:
\begin{exe}
    \ex
    \gll Einhver opnaði skápinn.\\
    somebody.\Third\Nom.\Sg{} open.\Third\Sg.\Pst{} cupboard.\Acc.\Sg.\Det\\
    \trans 'Somebody opened the cupboard.'
    \ex
    \gll Skápurinn var opnaður.\\
    cupboard.\Nom.\Sg.\M.\Det{} be.\Third\Sg.\Pst{} open.\Nom.\Sg.\M.\Pst.\Ptcp\\
    \trans 'The cupboard was opened.'
\end{exe}

Additionally we have the so-called middle forms, which are verb forms ending in \emph{-st}, that have a passive-like meaning:
\begin{exe}
    \ex
    \gll Skápurinn opnaðist.\\
    cupboard.\Nom.\Sg.\M.\Det{} open.\Nom.\Sg.\M.\Pst.\Ptcp\\
    \trans 'The cupboard opened.'
\end{exe}

There is no understood agent in \emph{-st} constructions, such that a sentence like this is ungrammatical in Icelandic:
\begin{exe}
    \ex
    \gll *Naglarnir framleiðast av Vírneti hf.\\
    nail.\Nom.\Sg.\Det{} produce.\Sg{} by.\textsc{prep} Virnet Inc.\\
\end{exe}

The \emph{perfect} in Icelandic is formed by the auxiliary \emph{hafa} 'have' an an uninflected past participle of the main verb,
as seen in the examples below. We will refer to the uninflected past participle as \emph{supine}, but it is identical to
the singular nominative/accusative of the participle:
\begin{exe}
    \ex
    \begin{xlist}
        \item \gll María hefur aldrei lesið þessa bók.\\
        Maria have.\Third\Sg.\Pst.\Aux{} never read.\textsc{sup} this.\Det{} book.\Nom.\Sg.\F{}\\
        \trans 'Maria has never read this book.'
        \item \gll Pósturinn hefur ekki komið í morgun.\\
        mail.\Sg.\Det{} have.\Third\Sg.\Pst.\Aux{} not come.\textsc{sup} in morning\\
        \trans 'The mail has not arrived this morning.'
        \item \gll Þessi bók hefur aldrei verið lesin.\\
        this book have.\Third\Sg.\Pst.\Aux{} never be.\textsc{sup} read.\Pst.\Ptcp.\Nom.\F{}\\
        \trans 'This book has never been read.'
    \end{xlist}
\end{exe}

In Icelandic a \emph{progressive} aspect is expressed by using the auxiliary \emph{vera} 'be' with the infinitive of the main verb
(including the infinitive marker \emph{að}):
\begin{exe}
    \ex
    \gll Ég var að borða morgunmatinn þegar hún kom.\\
    I be.\First\Sg.\Pst\Ind{} to eat.\Inf{} breakfast\Acc.\Det{} when she arrive.\Pst\\
    \trans 'I was eating breakfast when she arrived.'
\end{exe}


\subsection{Syntax}


We illustrate word order and clause structure in Icelandic by representing some sample
sentences from~\citep[17-18]{icelandic} as RRG trees. First we gloss the sentences, then we
draw the trees.

\begin{exe}
    \ex
    \begin{xlist}
        \item \gll Margir höfðu aldrei lokið verkefninu\\
        many.\Adv{} have.\Pst.\Aux{} never.\Adv{} finish.\Pst.\Ptcp{} assignment-\Det\\
        \trans 'Many had never finished the assignment'
        \item \gll Það höfðu aldrei margir lokið verkefninu\\
        there have never many.\Adv{} finish.\Pst.\Ptcp{} assignment-\Det\\
        \trans 'There were never many people who had finished the assignment'
        \item \gll hvort María hefði ekki lesið bókina\\
        whether Maria have not.\Neg{} read.\Pst.\Ptcp{} book-\Det\\
        \trans 'whether Maria has not read the book'
        \item \gll Bókina hefur hún ekki lesið\\     
        book-\Det{} have.\Pst.\Aux{} she not read\\
        \trans 'She has not read the book'
    \end{xlist}
\end{exe}


\begin{exe}
    \ex
    \begin{xlist}
    \item \psset{treesep=2ex}
        \CLAUSE{
        \lPERIPH{2}{AdvP}{\WORD{Margir}}
        \CORE{
            \WORD(AUX){höfðu}
            \WORD(AdvP){aldrei}
            \NUC{lokið}
            \ARG{\WORD(NP){verkefninu}}
        }
    }
    \dolinks
    \item \psset{treesep=2ex}
        \CLAUSE{
        \lPERIPH{2}{AdvP}{\WORD{Það}}
        \CORE{
            \WORD(AUX){höfðu}
            \WORD(AdvP){aldrei}
            \WORD(AdvP){margir}
            \NUC{lokið}
            \ARG{\WORD(NP){verkefninu}}
        }
    }
    \dolinks
    \item \psset{treesep=2ex}
        \CLAUSE{
        \lPERIPH{2}{AdvP}{\WORD{hvort}}
        \CORE{
            \WORD(AUX){María}
            \WORD(AdvP){hefði}
            \WORD(AdvP){ekki}
            \NUC{lesið}
            \ARG{\WORD(NP){bókina}}
        }
    }
    \dolinks
    \item \psset{treesep=2ex}
        \CLAUSE{
        \CORE{
            \ARG{\WORD(NP){Bókina}}
            \WORD(AdvP){hefur}
            \ARG{\WORD(PP){hún}}
            \WORD(AdvP){ekki}
            \NUC{lesið}
        }
    }
    \dolinks
    \end{xlist}
\end{exe}


\section{Faroese}

Following the overview of Icelandic morphology and syntax we take a similar look at Faroese.
We start with nominal categories before we move on to verbal categories.

\subsection{Nominal categories}

Like Icelandic, Faroese has three grammatical genders; masculine, feminine, and neuter.
Faroese nouns inflect for gender, but also number (singular/plural) and case (nominative/accusative/dative/genitive).
The inflectional endings vary according to gender. Definiteness of nouns is indicated by a suffixed article.

Just like nouns, adjectives also inflect for number, case, and gender. Additionally they also inflect for degree
(positive/comparative/superlative). Adjectives typically have two forms of inflections - strong or weak - depending
on the definiteness of the noun phrase they are a part of.

Finally, articles, pronouns, the cardinal numbers 1 to 3, and the ordinal numbers, inflect for number, case, and gender.

Faroese noun phrases show extensive agreement, including number agreement between nouns and the adjectives that modify them.
This holds for both attributive and predictive adjectives. We illustrate this using the noun \emph{drongur} 'boy'
with the adjective \emph{klókur} 'smart'~\citep[61]{faroese}:
\begin{exe}
    \ex
    \gll ein klókur drongur klókir drongir\\
    a smart boy smart boy\\
    \trans 'a smart boy' 'smart boys'
    \gll drongurin er klókur dreingirnir eru klókir\\
    boy-\Det{} is smart boy-\Det,\Pl{} is.\Pl{} smart.\Pl\\
    \trans 'the boy is smart' 'the boys are smart'
\end{exe}

The gender of nouns is reflected in the different forms of the personal pronouns used to refer to them and the
gender of adjectives and articles used to modify said nouns. We illustrate this with the following example:
\begin{exe}
    \ex
    \gll Hetta er ein klókur drongur\\
    this is a.\M{} smart.\M{} boy.\M\\
    Hann. er klókur\\
    he.\M{} is smart.\M\\
    \gll Hetta er ein klók genta\\
    this is a.\F{} smart.\F{} girl.\F\\
    Hon er klók\\
    she.\F{} is smart.\F{}\\
    \gll Hetta er eitt klókt barn\\
    this is a.\N{} smart.\N{} child.\N\\
    Tað er klókt\\
    it.\N{} is smart.\N\\
\end{exe}

Old Norse and older Faroes had four morphologically distinctive cases - nominative, accusative, dative, and genitive -
but only the three first are productive in modern spoken Faroese (modern \emph{written} Faroese still retains the genitive
to a degree). We illustrate this with a small example:
\begin{exe}
    \ex
    \gll Gentan svav\\
    girl-\Det.\Nom{} sleep.\Pst\\
    \trans 'The girl slept'\\
    \gll Eg sá gentuna\\
    I see.\Pst{} girl-\Det.\Acc\\
    \trans 'I saw the girl'\\
    \gll Hetta er hundurin hjá gentuni\\
    This is dog-\Det{} with girl-\Det.\Dat\\
    \trans 'This is the girl'{}s dog'
\end{exe}

Despite this a genitive form can be produced for nouns and personal pronouns, but less so for adjectives. The genitive form
of personal pronouns is widely used while the genitive form of many nouns is found in fixed expressions and as the first
part of certain compounds, but it is uncertain whether speakers intuitively interpret these forms as genitive~\citep[62]{faroese}.

Instead of genitive, modern spoken Faroese prefers prepositional constructions involving a dative form of the noun as
illustrated in these examples:
\begin{exe}
    \ex
    \gll Her eru húsini hjá einum ríkum manni\\
    here is.\Pl{} house-\Det.\Pl{} with a.\Dat{} rich.\Dat{} man.\Dat{}\\
    'Here is a rich man'{}s house/home'
    \gll Kettlingurin hjá kettuni hjá mær er vakur\\
    kitten-\Det{} with cat-\Det{}.\Dat{} with I.\Dat{} is beautiful\\
    'My cat'{}s kitten is beautiful'
\end{exe}

Above we saw examples with the preposition \emph{hjá} 'with', but Faroese speakers also use other prepositions with the dative,
depending on the semantic function. We'll look at some examples~\citep{faroese}:
\begin{exe}
    \ex
    \gll takið á húsinum motorurin í bilinum\\
    roof-\Det{}.\Nom.\N{} on house-\Det{}.\Dat{}.\N{} motor-\Det{}.\Nom{}.\M{} in car-\Det.\Dat.\M\\
    \trans 'the roof of the house' 'the car'{}s engine'
    \gll abbi at dreinginum aldurin á kirkjuni\\
    grandfather to boy-\Det.\M.\Dat{} age-\Det on church-\Det.\Dat\\
    \trans 'the boy'{}s grandfather' 'the age of the church' 
    \gll halin á kúnni tenninar í hundinum\\
    tail-\Det on cow-\Det.\Dat{} tooth-\Det.\Pl{} in dog-\Det.\Dat\\
    \trans 'the cow'{}s tail' 'the dog'{}s teeth'
    \gll høvdið á mær eyguni í honum\\
    head-\Det on I.\Dat{} eyes-\Det{} in he.\Dat\\
    \trans 'my head' 'his eyes'
\end{exe}

With nouns denoting family relationships an accusative form is normally used instead of genitive or a prepositional phrase
as illustrated here:
\begin{exe}
    \ex
    \gll pápi dreingin mamma gentuna beiggi Jógvan\\
    father boy-\Det.\Acc{} mum girl-\Det.\Acc{} brother Jógvan.\Acc\\
    \trans 'the boy'{}s father' 'the girl'{}s mother' 'Jógvan'{}s brother'
\end{exe}

As mentioned above, adjectives can be grouped into two categories: \emph{strong} adjectives and \emph{weak} adjectives.
Which category they belong to depends on the definiteness of the noun phrase they form a part of. The general rule is that
the adjective takes the weak form if the noun phrase is definite, and the strong form if the noun phrase is indefinite.
We illustrate this with two examples~\citep[65]{faroese}:
\begin{exe}
    \ex
    \begin{xlist}
        \item \gll Hetta er ein stórur bilur og ein lítil bók.\\
        this is a big.\Nom.\Sg.\M{} car.\Nom.\Sg.\M{} and a small.\Nom.\Sg.\F{} book.\Nom.\Sg.\F\\
        \trans 'This is a big car and a small book'
        \item \gll Hetta er tann stóri bilurin og tann lítla bókin.\\
        this is the big.\Nom.\Sg.\M{} car-\Det.\Nom.\Sg.\M{} and the small-\Det.\Nom.\Sg.\F{} book\\
        \trans 'This is the big car and the small book'
    \end{xlist}
\end{exe}

Finally, most adjectives can be inflected for degree by adding the suffixes \emph{-(a)r} and \emph{-(a)st} in
comparative and superlative, respectively. Indeclinable adjectives express difference in degree by using the
auxiliary verbs \emph{meiri} 'more' (comparative) and \emph{mest} 'most' (superlative). We'll finish by showing a
couple of examples of this:
\begin{exe}
    \ex
    \begin{xlist}
        \item \gll gulur gul-a-ri gul-ast-ur\\
        yellow yellower yellowest\\
        \item \gll hóskandi meiri hóskandi mest hóskandi\\
        appropriate more appropriate most appropriate\\
    \end{xlist}
\end{exe}


\subsection{Verbal categories}

Faroese verbs are inflected by person, number, and tense, with the two following characteristics with respec to
person inflection~\citep[67]{faroese}:
\begin{enumerate}
    \item Faroese verbs do not show any person distinctions in the plural and regular (weak).
    \item Faroese verbs do not show any person distinctions neither in the singular nor in the past tense.
\end{enumerate}

This can be illustrated with the following example:
\begin{exe}
    \ex
    \begin{xlist}
        \item \gll eg kalli tú kallar hann/hon/tað kallar\\
        I call.\First\Sg.\Prs{} you call.\Second\Sg.\Prs{} he/she/it call.\Third\Sg.\Prs\\
        \trans 'I call' 'you call' 'he/she/it calls'
        \item \gll vit kalla tit kalla teir/tær/tey kalla\\
        we call.\First\Pl.\Prs{} you.\Pl{} call.\Second\Pl.\Prs{} they.\M/\F/\N{} call.\Third\Pl.\Prs\\
        \trans 'we call' 'you call' 'they call'
        \item \gll eg kallaði tú kallaði hann/hon/tað kallaði\\
        I call.\First\Sg.\Pst{} you call.\Second\Sg.\Pst{} he/she/it call.\Third\Sg.\Pst\\
        \trans 'I called' 'you called' 'he/she/it called'
    \end{xlist}
\end{exe}

Faroese has two distinct imperative forms, plural and singular, as illustrated here:
\begin{exe}
    \ex
    \gll Gev/gevið hesum manninum gætur!\\
    give.\Sg/\Pl{} this man-\Det.\Third\Sg\Dat{} attention\\
    \trans 'Give attention to this man!'
\end{exe}

The singular \emph{gev} would be used if addressing one person, and the plural \emph{gevið} if more than one person
is being addressed. There was no distinctive plural imperative in Old Norse, where the \Second\Pl{} indicative had this role,
but in Faroese there is a distinction between the default finite forms and the imperative forms. We illustrate this below,
where the non-imperative forms are referred to as \emph{indicative}, although it is uncertain that one can speak of indicative
in Faroese, because there is no productive contrasting subjunctive in Faroese~\citep[67-68]{faroese}:
\begin{exe}
    \ex
    \begin{xlist}
        \item \gll Tú fert til hús.\\
        you.\Second\Sg{} go.\Second\Sg.\Prs.\Ind{} to house\\
        \trans 'You go home'
        \item \gll Far til hús!\\
        go.\Second\Sg.\Imp{} to house\\
        \trans 'Go home!'
        \item \gll Tit fara til hús.\\
        you.\Second\Pl{} go.\Second\Pl.\Prs.\Ind{} to house\\
        \trans 'You go home'
        \item \gll Farið til hús!\\
        go.\Second\Pl.\Imp{} to house\\
        \trans 'Go home!'
    \end{xlist}
\end{exe}

We mentioned above that the subjunctive is not productive in Faroese anymore, and it should be added that only
a few relic forms exist in main clauses in relatively fixed expressions and in religious language. And where they
exist they almost exclusively express \emph{optative modality}, as illustrated below~\citep[68]{faroese}:
\begin{exe}
    \ex
    \begin{xlist}
        \item \gll Jesus fylgir tær\\
        Jesus follow.\Third\Sg.\Ind{} you\\
        \trans 'Jesus is with you'
        \item \gll Jesus fylgi tær\\
        Jesus follow.\Third\Sg.\Subj{} you\\
        \trans 'Jesus be with you' 
        \item \gll Gud signar Føroyar\\
        God bless.\Third\Pl.\Ind{} Faroes\\
        \trans 'God blesses the Faroes'
        \item \gll Gud signi Føroyar\\
        God bless.\Third\Pl.\Subj{} Faroes\\
        \trans 'God bless the Faroes'
    \end{xlist}
\end{exe}

Examples of other, relatively fixed optative forms, which aren't religious expressions, are:
\begin{exe}
    \ex
    \begin{xlist}
        \item \gll Hann leingi livi!\\
        He long live.\Third\Sg.\Subj{}\\
        \trans 'Long live he!'
        \item \gll Gævi at tað skjótt varð heystfrí!\\
        Give.\Third\Sg.\Pst.\Subj{} that it soon become.\Sg.\Pst{} {autumn break}\\
        \trans 'I wish we had autumn break soon!'
        \item \gll Hevði tað nú bara gingist henni væl.\\
        Have.\Third\Sg.\Pst.\Subj{} it now just go.\Sg.\Pst{} her well\\
        \trans 'I wish things would go well for her'
    \end{xlist}
\end{exe}

Of these the first one uses the present subjunctive while the two others use the past subjunctive (or what was the
past subjunctive in older Faroese).

Typically the passive is formed with the auxiliary verbs \emph{verða} 'be, become' and \emph{blíva} 'be, become'. 
The participle agree in case, gender, and number with a nominative subject, and the agent is more frequently left
out than in English. If the agent is included, it is with the auxiliary \emph{av} 'by', which takes a dative form.
We illustrate this with some examples~\citep[69]{faroese}:
\begin{exe}
    \ex
    \begin{xlist}
        \item \gll Hann kysti hana\\
        he.\Third\Sg.\Nom{} kiss.\Sg.\Pst{} her.\Third\Sg\Acc\\
        \trans 'He kissed her'
        \item \gll Hon varð/bleiv kyst (av honum)\\
        she.\Third\Sg.\Nom{} be.\Sg.\Pst.\Aux{} kiss.\Nom.\Sg.\F.\Pst.\Ptcp{} (by him.\Dat)\\
        \trans 'She was kissed by him'
    \end{xlist}
    \ex
    \begin{xlist}
        \item \gll Hon kysti teir\\
        she.\Third\Sg\Nom{} kiss.\Sg.\Pst.\Ind{} them.\Acc.\Pl.\M{}\\
        \trans 'She kissed them'
        \item \gll Teir vórðu/blivu kystir (av henni)\\
        they.\Third\Pl.\Nom.\M{} be.\Pl.\Pst.\Aux{} kiss.\Nom.\Pl.\M{} (by her.\Dat)\\
        \trans 'They were kissed by her'
    \end{xlist}
    \ex
    \begin{xlist}
        \item \gll Teir smurdu hann av\\
        they smear he.\Third\Sg.\Acc{} off\\
        \trans 'They beat him up'
        \item \gll Hann varð/bleiv avsmurdur\\
        he.\Third\Sg.\Nom{} be offsmear.\Nom.\Sg.\M.\Pst.\Ptcp\\
        \trans 'He was beaten up'
    \end{xlist}
\end{exe}

In Faroese it is frequently possible to form so-called \emph{-st}-forms, or \emph{middle} forms, by adding the suffix
\emph{-st} to various inflectional forms of the verb. The meaning of the middle forms varies widely in Faroese, but
the usages most frequently mentioned in discussions of the middle forms are reflexive, reciprocal, or passive.
We give some examples below:
\begin{exe}
    \ex
    \begin{xlist}
        \item \gll Eg settist niður\\
        I sit down\\
        \trans 'I sat down' (reflexive meaning)
        \item \gll Teir berjast altíð\\
        they fight always\\
        \trans 'They always fight' (reciprocal meaning)
        \item \gll Oyggin kallast Nólsoy\\
        island-\Det{} call Nólsoy\\
        \trans 'The island is called Nólsoy' (passive meaning)
    \end{xlist}
\end{exe}

Perfect tense is either formed with the auxiliary \emph{hava} 'have' and the supine (\Sg\N{} of \Pst\Ptcp{}) of the main verb,
or it is formed with the auxiliary \emph{vera} 'be' and the inflected and agreeing past participle. \emph{Hava} is used
with all transitive verbs and most intransitive verbs~\citep[72]{faroese}:
\begin{exe}
    \ex
    \begin{xlist}
        \item \gll Hon hevur lisið bókina.\\
        she have.\Third\Sg.\Pst{} read.\textsc{sup} book-\Det.\Acc\\
        \trans 'She has read the book'
        \item \gll Teir hava sovið leingi.\\
        they have.\Third\Pl.\Pst{} sleep.\textsc{sup} long\\
        \trans 'They have slept for long'
        \item \gll Hann hevur verið ríkur.\\
        he have\Third\Sg.\Pst{} be.\textsc{sup} rich\\
        \trans 'He has been rich'
        \item \gll Hann er vorðin ríkur.\\
        he.\Nom.\Sg.\M{} is become.\Nom.\Sg.\M{} rich\\
        \trans 'He has become rich'
    \end{xlist}
\end{exe}

Past perfect is formed with past tense of the relevant auxiliary (\emph{vera/hava}as mentioned above), and the perfect
passive is formed with the auxiliary \emph{vera} 'be', not \emph{hava} 'have'. We illustrate this with a couple of examples:
\begin{exe}
    \ex
    \begin{xlist}
        \item \gll Hann hevði verið ríkur.\\
        he have.\Pst{} be.\textsc{sup} rich.\Nom.\M{}\\
        \trans 'He had been rich'
        \item \gll Hann var vorðin ríkur.\\
        he be.\Third\Sg.\Pst{} become.\Pst.\Ptcp{} rich.\Nom.\M{}\\
        \trans 'He had become rich'
    \end{xlist}
    \ex
    \gll Hann er/*hevur ofta vorðin/blivin avsmurdur.\\
    he be.\Third\Sg.\Prs{}/*have often become.\Pst.\Ptcp{} off-smear.\Nom.\M{}\\
    \trans 'He has often been beaten up'
\end{exe}

The indicative-subjunctive distinction is not productive in Faroese and past subjunctive forms generally do not exist.
Past subjunctive was commonly used in Old Norse (and still is in Modern Icelandic) to indicate a counterfactual or hypthetical
situation. The regular past tense can have this function is Faroese, but the meaning of such forms are typically
ambiguous. We illustrate this with a few examples:
\begin{exe}
    \ex
    \begin{xlist}
        \item \gll Eg gjørdi tað fegin.\\
        I do.\Pst{} it gladly\\
        \trans 'I did it gladly.' or 'I would gladly do it.'
        \item \gll Hann hevði dripið hundin.\\
        he have.\Pst{} kill.\Pst.\Ptcp{} dog-\Det.\Acc.\M{}\\
        \trans 'He had killed the dog.' or 'He would have killed the dog.'
        \item \gll Hann tók bókina.\\
        he take.\Pst{} book-\Det.\Nom.\F{}\\
        \trans 'He took the book.' or 'He would gladly take the book if...'
        \item \gll Eg hevði fegin gjørt tað, um eg fekk pengar fyri tað.\\
        I have.\Pst{} gladly do.\Pst.\Ptcp{} it if I get.\Pst{} money for it\\
        \trans 'I would gladly have done it if I was paid for it.'
        \item \gll Hann drap hundin, um hann fekk hendur á honum.\\
        he kill.\Pst{} dog-\Det{} if he get.\Pst{} hand.\Nom.\Pl. on it\\
        \trans 'He would kill the dog if he got his hands on it.'
        \item \gll Hann hevði tikið bókina frá mær, um hann hevði sæð meg lisið í henni.\\
        he have.\Pst{} take.\Pst.\Ptcp{} book-\Det{}.\Nom{} from me if he have.\Pst{} see.\Pst.\Ptcp{} me read.\Pst.\Ptcp{} in it\\
        \trans 'He would have taken the book from me if he had seen me reading it.'
    \end{xlist}
\end{exe}

\subsection{Syntax}

The default word order in Faroese is subject-verb-object (SVO) or subject-auxiliary-main verb-object (SAVO), both in main clauses
and embedded ones. We look at some examples~\citep[236]{faroese}:
\begin{exe}
    \ex
    \begin{xlist}
        \item \gll Jógvan las bókina.\\
        Jógvan read.\Pst{} book-\Det.\Acc.\Sg.\F{}\\
        \trans 'Jógvan read the book.'
        \item \gll Jógvan hevur lisið bókina.\\
        Jógvan have.\Pst{} read.\Pst.\Ptcp{} book-\Det.\Acc.\Sg.\F{}\\
        \trans 'Jógvan has read the book.'
        \item \gll Eg haldi, at Jógvan hevur lisið bókina.\\
        I think.\Prs{} that Jógvan have.\Pst{} read.\Pst.\Ptcp{} book-\Det\\
        \trans 'I think that Jógvan has read the book.'
    \end{xlist}
\end{exe}

As a rule, the indirect object precedes the direct object and typically appears in the dative, although
indirect objects in the accusative form also appear. Lets look at some examples:
\begin{exe}
    \ex
    \begin{xlist}
        \item \gll Turið gav Hjalmari nógvar bøkur.\\
        Turið.\Nom{} give.\Pst{} Hjalmar.\Dat{} many book.\Nom.\Pl\\
        \trans 'Turið gave Hjalmar many books.'
        \item \gll Eg spurdi, um Zakaris seldi Eivindi tann gamla bilin.\\
        I ask if Zakaris.\Nom{} sell.\Pst{} Eivind.\Dat{} the old car-\Det.\Acc\\
        \trans 'I asked if Zakaris sold the old car to Eivind.'
        \item \gll Hon lærdi meg niðurlagið.\\
        she teach.\Pst{} me.\Acc refrain-\Det.\Acc\\
        \trans 'She taught me the refrain.'
    \end{xlist}
\end{exe}

Should we move an object or a prepositional phrase, or some other non-subject, to the front of a sentence, as is done
in Topicalisation, the
finite verb shows up in second place followed by the subject, i.e. Faroese is a "verb-second" (V2) language like the
other Germanic languages except English. We take a look at some examples~\citep[238-239]{faroese}:
\begin{exe}
    \ex
    \begin{xlist}
        \item \gll \emph{Hesa bókina} hevur Jógvan lisið.\\
        {this book}-\Det.\Acc{} have.\Pst{} Jógvan.\Nom{} read.\Pst.\Ptcp{}\\
        \trans 'This book has Jógvan read.'
        \item \gll \emph{Tann gamla bilin} seldi Zakaris Eivindi.\\
        {the old car}-\Det.\Acc{} sell.\Third.\Sg.\Pst{} Zakaris.\Nom{} Eivind.\Dat.\\
        \trans 'The old car Zakaris sold to Eivind.'
        \item \gll \emph{Jóannes} haldi eg eigur hesa bókina.\\
        Jóannes.\Nom{} think I own.\Third\Sg.\Prs{} this book-\Det.\Acc\\
        \trans 'Jóannes, I think, owns this book.'
    \end{xlist}
\end{exe}

As is typical for modern Germanic languages, adjectives precede the noun they modify:
\begin{exe}
    \ex
    \begin{xlist}
        \item \gll ein \emph{vøkur} genta, ein \emph{bláur} bilur\\
        a beautiful.\Nom.\F{} girl.\Nom.\F{}, a blue\Nom.\M{} car.\Nom.\M{}\\
        \trans 'a beautiful girl', 'a blue car'
        \item \gll Hann kom súkklandi á einari \emph{gamlari} súkklu.\\
        he come.\Pst{} ride.\Prs.\Ptcp{} on an old.\Dat.\F{} bicycle.\Dat.\F{}\\
        \trans 'He came riding on an old bicycle.'
    \end{xlist}
\end{exe}

When it comes to adverbs we can broadly distinguish between three basic adverbial positions~\citep[241]{faroese}:
\begin{enumerate}
    \item the medial position, following the finite verb
    \item the verb phrase position, following a possible object and other elements of the verb phrase
    \item the modifying position, when the adverb is modifying and adjective and other adverbs
\end{enumerate}

We illustrate this with a few examples:
\begin{exe}
    \ex
    \begin{xlist}
        \item \gll Tey hava \emph{ikki/ivaleyst/jú/aldri} lisið bókina.\\
        they have not/undoubtedly/actually/never.\Adv{} read book-\Det.\Acc\\
        \trans 'They have not/undoubtedly/actually/never read the book.'
        \item \gll Tey hava lisið bókina \emph{tá/har/væl og virðiliga}.\\
        they have read book-\Det.\Acc{} {then/there/well and thoroughly}.\Adv{}\\
        \trans 'They have read the book then/there/well and thoroughly.'
        \item \gll Tey hava lisið hesa ógvuliga longu bókina sera væl.\\
        they have read this extremely.\Adv{} long book-\Det.\Acc{} very.\Adv{} well\\
        \trans 'They have read this extremely long book very well.'
    \end{xlist}
\end{exe}

Let's end by looking at the case of subject, object, and indirect object. The regular \emph{subject} case in Faroese
is \emph{nominative}, as we already have seen in previous examples:
\begin{exe}
    \ex
    \begin{xlist}
        \item \gll \emph{Hann} skrivar.\\
        he.\Nom.\Sg{} write.\Third\Sg\\
        \trans 'He writes.'
        \item \gll \emph{Hon} arbeiðir.\\
        she.\Nom.\Sg{} work.\Third\Sg\\
        \trans 'She works.'
        \item \gll \emph{Børnini} spæla.\\
        child-\Det.\Nom.\Pl play.\Third\Pl\\
        \trans 'The children play'
        \item \gll \emph{Vit} settu niður epli {í gjár}.\\
        we.\Nom.\Pl{} put down potato.\Pl{} yesterday\\
        \trans 'We planted potatoes yesterday.'
    \end{xlist}
\end{exe}

Before we continue, it should be mentioned that some verbs take non-nominative subjects in modern Faroese,
but we won't look further into that here.

Looking at \emph{direct object} case we can say that the \emph{accusative} is the default case,
as we have already seen in previous examples:
\begin{exe}
    \ex
    \begin{xlist}
        \item \gll Hon keypti bókina.\\
        she buy.\Pst{} book-\Det.\Acc\\
        \trans 'She bought the book.'
        \item \gll Hann seldi telduna.\\
        he sell.\Pst{} computer-\Det.\Acc\\
        \trans 'He sold the computer.'
    \end{xlist}
\end{exe}

It should be mentioned, however, that a number of verbs take a direct object in dative case, which we also touched upon earlier,
but we will restrict ourselves here to saying that among the verbs that take a direct object in the dative are
verbs of helping, ordering, praising, thanking, welcoming, etc., leading one to associate the dative objects with
thematic roles such as recipients and experiencers. The semantic verb classes and thematic roles we have mentioned are roughly
the same as the ones taking a dative object in Icelandic, although dative objects are getting more rare in modern Faroese
while still being prevalent in Icelandic~\citep[257-258]{faroese}.

At last we have the \emph{indirect object}, which takes the \emph{dative} as its default case, as we have seen in previous examples.
Among the verbs taking an indirect object are verbs meaning 'sell', 'lend', 'give', 'send', etc. Let's look at three examples:
\begin{exe}
    \ex
    \begin{xlist}
        \item \gll Hann beyð henni starv.\\
        he offer.\Pst{} she.\Dat{} job.\Acc\\
        \trans 'He offered her a job.'
        \item \gll Tey fingu sær bil.\\
        they get.\Pst{} themselves.\Dat{} car.\Acc\\
        \trans 'They got themselves a car.'
        \item \gll Fyrigev honum syndir hansara.\\
        forgive.\Imp{} him.\Dat{} sin.\Pl.\Acc{} he.\Gen\\
        \trans 'Forgive him his sins.'
    \end{xlist}
\end{exe}

We notice from the examples that these verbs, \emph{bjóða} 'offer', \emph{fáa} 'get', \emph{fyrigeva} 'forgive', are
examples of verbs that take two objects, one dative (indirect) object and one accusative (direct) object. We call these verbs \emph{ditransitive}.
This dative--accusative pattern is also the most common case marking pattern of ditransitive verbs in Icelandic, but there
we also have many other patterns, such as dative--dative, dative--genitive, accusative--dative, accusative--genitive, and
accusative--accusative, most of which are not found in Faroese~\citep[262-263]{faroese}.


\section{Conclusion}

In the preceding sections we have gone through the morphology and syntax of Icelandic and Faroese, respectively.
We have seen that it is possible to compare the complexity of two languages by looking at the structure of
equivalent phenomena.

We have seen that inflection of nouns is similar in both languages; both have four cases (\Nom, \Acc, \Dat, \Gen{}, although genitive
is largely not productive in Faroese) and three grammatical genders (\M{},\F{},\N{}). On the other hand,
verb inflections are more complex in Icelandic than Faroese. Firstly, the distinction between subjunctive and
indicative has been lost in Faroese (except for a few fixed expressions), secondly the distinction
between first, second, and third person in the plural has also been lost in Faroese, as has the
person distinction in the past tense singular of weak verbs. Despite this, an innovation in faroese
is the occurrence of the plural imperative, which did not exist on Old Norse, and doesn't in Modern Icelandic either,
where the \Second\Pl{} indicative has this role.

\subsection{Future work}

In the future this comparison could be built out to emphasise syntax more and put the analysis
into the context of Role and Reference Grammar (RRG), which would nicely lay the groundwork for building of machine parsers
for Icelandic and Faroese, respectively. We showed a sample of RRG trees for Icelandic, but this
should be done for Faroese as well.
\\\\



    
\bibliographystyle{lin-v2/lin}    
\bibliography{references}


\end{document}